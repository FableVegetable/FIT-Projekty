\documentclass[11pt, a4paper, twocolumn ]{article}

\usepackage[czech]{babel}
\usepackage[utf8]{inputenc}
\usepackage{times}

\usepackage[left=2cm, text={17cm, 24cm}, top=2.5cm]{geometry}
\newcommand{\myuv}[1]{\quotedblbase #1\textquotedblleft}

\title{Typografie a publikování \\1. projekt}
\author{Tomáš Zubrik \\xzubri00@stud.fit.vutbr.cz}
\date{}

\begin{document}

\maketitle

\section{Hladká sazba}
Hladká sazba je sazba z~jednoho stupně, druhu a~řezu pí­sma sázená na stanovenou šířku odstavce. Skládá se z~odstavců, které obvykle začínají zarážkou, ale mohou být sázeny i~bez zarážky -- rozhodující je celková grafická úprava. Odstavce jsou ukončeny východovou řádkou. Věty nesmějí začínat číslicí.

Barevné zvýraznění, podtrhávání slov či různé velikosti písma vybraných slov se zde také nepoužívá. Hladká sazba je určena především pro delší texty, jako je například beletrie. Porušení konzistence sazby působí v~textu rušivě a~unavuje čtenářův zrak.

\section{Smíšená sazba }
Smíšená sazba má o~něco volnější pravidla než hladká sazba. Nejčastěji se klasická hladká sazba doplňuje o~další řezy písma pro~zvýraznění důležitých pojmů. Existuje \myuv{pravidlo}:

\begin{quotation}
Čím více {\textbf{druhů}}, {\textbf{\textit{řezů}}}, {\scriptsize{velikostí}}, barev pís\-ma a~jiných efektů použijeme, tím{\textit{ profesionálněji}} bude  dokument vypadat. Čtenář tím bude vždy {\Huge{nadšen!}}
\end{quotation}

\textsc{Tímto pravidlem se \underline{nikdy} nesmíte řídit}. Příliš časté zvýrazňování textových elementů a~změny velikosti {\tiny{ písma}} jsou {\LARGE{známkou}} {\huge{\textbf{a\-ma\-té\-ris\-mu}}} autora a~působí {\textit{\textbf{velmi}} {\textit{rušivě}}. Dobře navržený dokument nemá obsahovat více než 4 řezy či druhy písma. {\texttt{Dobře navržený dokument je decentní, ne chaotický.}}

Důležitým znakem správně vysázeného dokumentu je konzistentní používání různých druhů zvýraznění. To například může znamenat, že {\textbf{tučný řez}} písma bude vyhrazen pouze pro~klíčová slova, {\textit{skloněný řez}} pouze pro doposud neznámé pojmy a nebude se to míchat. Skloněný řez nepůsobí tak rušivě a~používá se častěji. V~\LaTeX u jej sázíme raději příkazem \verb|\emph{text}| než \verb|\textit{text}|.

Smíšená sazba se nejčastěji používá pro sazbu vědeckých článků a~technických zpráv. U~delších dokumentů vědeckého či technického charakteru je zvykem upozornit čtenáře na~význam různých typů zvýraznění v~úvodní kapitole.

\section{České odlišnosti}
Česká sazba se oproti okolnímu světu v~některých aspektech mírně liší. Jednou z~odlišností je sazba uvozovek. Uvozovky se v~češtině používají převážně pro~zobrazení přímé řeči. V~menší míře se používají také pro~zvýraznění přezdívek a~ronie. V~češtině se používá tento {\textbf{\myuv{typ uvozovek}}} namísto anglických "uvozovek". Lze je sázet připravenými příkazy nebo při~použití UTF-8 kódování i~přímo.

Ve~smíšené sazbě se řez uvozovek řídí řezem prvního uvozovaného slova. Pokud je uvozována celá věta, sází se koncová tečka před uvozovku, pokud se uvozuje slovo nebo část věty, sází se tečka za~uvozovku.

Druhou odlišností je pravidlo pro sázení konců řádků. V~české sazbě by řádek neměl končit osamocenou jednopísmennou předložkou nebo spojkou. Spojkou {\myuv{a}} končit může při~sazbě do~25 liter. Abychom \LaTeX u zabránili v sázení osamocených předložek, vkládáme mezi předložku a slovo {\textbf{nezlomitelnou mezeru}} znakem {\verb|~|} (vlnka, tilda). Pro~automatické doplnění vlnek slouží­ volně šiřitelný program \textit{vlna} od pana~Olšáka{\footnote{Viz http://petr.olsak.net/ftp/olsak/vlna/.}}.

\end{document}
