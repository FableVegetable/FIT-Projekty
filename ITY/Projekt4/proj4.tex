\documentclass[11pt,a4paper,titlepage]{article}
\usepackage[left=2cm,text={17cm,24cm},top=3cm]{geometry}
\usepackage[czech]{babel}
\usepackage[utf8]{inputenc}
\usepackage{url}
\bibliographystyle{czplain}
\usepackage{times}

\begin{document}

\begin{titlepage}
\begin{center}
    {\textsc{\Huge Vysoké učení technické v~Brně}}\\
    \smallskip
    {\huge\textsc{Fakulta informačních technológií}}\\
    \bigskip
    \vspace{\stretch{0.382}}    
    \LARGE{Typografie a~publikování\,--\,4.\,projekt}\\
    \smallskip
    \Huge{Bibliografické citace}\\
    \vspace{\stretch{0.618}}
\end{center}
    {\Large 2017 \hfill Tomáš Zubrik }
\end{titlepage}

\section{Úvod}
Vývoj osobných počítačov výrazne poznamenal elektronické spracovanie textu. Počítačmi sme schopní rýchlo a~efektívne vytvoriť to, čo skôr vznikalo použitím zložitej technológie a~veľkej spotreby materiálu. Vznikli rady firiem a~aplikačných programov na~prácu a~zpracovanie textu, ktoré sa stále vyvíjajú a~zdokonaľujú. Hovoríme o~\textit{typografii}.
\cite{Rybicka}

\section{Typografia}
Typografia má v~prvom rade komunikačnú a~estetickú funkciu, ale taktiež aj umeleckú. Typografiu ovplyvnili príbuzné odbory ako kaligrafia, ilustrácie a~rozličné grafické techniky.\cite{Hoskova}
Typografia utvára a~vplýva na najmenej dva druhy zmyslov. Má vizuálny a~historický zmysel. Vizuálna stránka typografie je ľahko dostupná. Materiály pre štúdium jej vizuálnej podoby sú rozšírené a~je ich mnoho, ale historický aspekt typografie je do značnej miery skrytý, ak nemáme k dispozícii štúdium rukopisov, starovekých nápisov a~kníh.\cite{Bringhurst} 

\section{Písmo}
Písmo prešlo dlhým a~zložitým vývojom. Dnešné písma, ktoré sa bežne používajú v~Európe a~v~Amerike boli odvodené od písem, ktoré vznikli pred pár tisíc rokmi v~Egypte a~Mezopotánii. Písmo sa vývijalo mnoho stáročí, počnúc egyptským znakovým písmom, cez grécku abecedu, rímsku latinku až po dnešné písma. S~výnalezom knihtlače sa objavujú nové písma, ktoré často nesú meno podľa svojho tvorcu.  \cite{Cerny}

\section{Štandardizácia}
Typografia ako iné odbory ľudskej činnosti potrebujú štandardizáciu, prečo? Na polygrafickom procese sa podieľajú grafici, kooperanti, zadávatelia, ktorí majú rôzne zvyklosti, úroveň a povedomie o svojej práci.  Spracovanie zakázok je zvyčajne pod časovým tlakom a do procesu vstupuje celá rada hardwaru a softwaru.\cite{Zapotocky}

\section{Obaly predávajú}
Obal knihy alebo tlačoviny je pre ľudí veľmi dôležitý, lebo si utvárame prvý dojem o~danej veci. Posudzujeme kvalitu výrobku a~obsahu bez znalosti vnútornej štruktúry. Grafický design a~grafické spracovanie obalu je veľmi doležité a~vplýva na naše zmysly.\cite{Sedlacek}

Význam obalov a~grafického designu si môžme všimnúť aj na časopisoch, ktoré sa venujú typgrafii. \cite{Digirama}.

\section{\LaTeX}
Systémy \TeX a~\LaTeX sú systémy určené pre profesionálne a~poloprofesionálne sádzanie dokumentov. \LaTeX je systém maker vystavaný nad \TeX om. Jazyk \TeX je veľmi mocný jazyk, ale ukázalo sa, že pre bežné používanie je príliš zložitý, a~preto sa z~neho odvodil šablónový a~značne jednoduchší systém \LaTeX , ktorý je pre bežné používanie oveľa vhodnejší. \cite{Martinek}
\LaTeX je ideálné riešenie v~prípade výroby normalizovaných dokumentov ako sú články, správy alebo knihy. Tieto typy dokumentov umožňuje upravovať pohodlne a~spoľahlivo. Nevýhoda \LaTeX u je, že užívateľ sa ho musí naučiť, aby ho mohol efektívne využiť.\cite{Sopuch}

\newpage
\section{Záver}
Prejsť z~Wordu alebo z~Writeru na \LaTeX určite nie je jednoduché, ale určite sa to oplatí.  \LaTeX neobmedzuje ako tieto textové editory. Máme neobmedzené možnosti tvorby vektorovej grafiky, formátovania a vysádzania textu.~\cite{Found}
Typografia a~\LaTeX \ sú v~dnešnom svete  zaujímavé odvetvia a~ich výhody a~možnosti sú len výhodou, či uvažujeme o~tvorbe bakalárskej alebo diplomovej práce, alebo o~profesionálnom zameraní a~tvorbe štruktúrovaných dokumentov.
\newpage
\renewcommand{\refname}{Referencie}
\bibliography{odkazy}

\end{document}
